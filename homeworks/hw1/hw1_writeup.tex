% --------------------------------------------------------------
% This is all preamble stuff that you don't have to worry about.
% Head down to where it says "Start here"
% --------------------------------------------------------------
 
\documentclass[12pt]{article}
 
\usepackage[margin=1in]{geometry} 
\usepackage{amsmath,amsthm,amssymb}
\usepackage{listings}
\usepackage{bm}

 
\newcommand{\N}{\mathbb{N}}
\newcommand{\Z}{\mathbb{Z}}
 
\newenvironment{theorem}[2][Theorem]{\begin{trivlist}
\item[\hskip \labelsep {\bfseries #1}\hskip \labelsep {\bfseries #2.}]}{\end{trivlist}}
\newenvironment{lemma}[2][Lemma]{\begin{trivlist}
\item[\hskip \labelsep {\bfseries #1}\hskip \labelsep {\bfseries #2.}]}{\end{trivlist}}
\newenvironment{exercise}[2][Exercise]{\begin{trivlist}
\item[\hskip \labelsep {\bfseries #1}\hskip \labelsep {\bfseries #2.}]}{\end{trivlist}}
\newenvironment{reflection}[2][Reflection]{\begin{trivlist}
\item[\hskip \labelsep {\bfseries #1}\hskip \labelsep {\bfseries #2.}]}{\end{trivlist}}
\newenvironment{proposition}[2][Proposition]{\begin{trivlist}
\item[\hskip \labelsep {\bfseries #1}\hskip \labelsep {\bfseries #2.}]}{\end{trivlist}}
\newenvironment{corollary}[2][Corollary]{\begin{trivlist}
\item[\hskip \labelsep {\bfseries #1}\hskip \labelsep {\bfseries #2.}]}{\end{trivlist}}
 
\begin{document}
 
% --------------------------------------------------------------
%                         Start here
% --------------------------------------------------------------
 
%\renewcommand{\qedsymbol}{\filledbox}
 
\title{Homework 1}%replace X with the appropriate number
\author{Guozhen Li\\ %replace with your name
STA 208 - Statistical Machine Learning} %if necessary, replace with your course title
 
\maketitle


\section{Exercise 1}
\subsection{Predictor minimizing true risk}
The true risk is
\begin{align*}
	R(g) & = \mathbb{E}[\ell(Y,g(X))]
\end{align*}
By conditioning on $X$, we can write $R(g)$ as
\begin{align*}
	R(g) & = \mathbb{E}_X \ell(Y, g(X)) \cdot Pr(Y|X)
\end{align*}
Minimize $R(g)$ pointwise:
\begin{align*}
	\hat{g}(x) = \arg\min_{h\in\{0,1\}} \ell(Y, h) Pr(Y|X=x)
\end{align*}
With $\ell()$ being the Hamming loss function this simplifies to:
\begin{align*}
	\hat{g}(x) = \arg\max_{h\in\{0,1\}} Pr(Y=h|X=x)
\end{align*}
* Reference: ESL pg. 20

\subsection{True risk of Bayes classifier}
???

\subsection{Classify with a single $x_j$}
To fit a model like $h(x)=1\{x_j>0\}$,
we can try all possible $j$'s and find the one that gives smallest emperical risk.
Pseudo code for this algorithm as follows:
\begin{lstlisting}[language=Python]
	best_R = Inf
	best_j = None
	for j in [1, 2, ..., p]:
		pred = [1 if x[i,j]>0 else 0 for i in 1:N]
		loss = xor(pred, y)
		risk = sum(loss)/N
		if risk < best_R:
			best_R = risk
			best_j = j
	return best_j
\end{lstlisting}
This algorithm returns the best $j$.

When making a prediction based on a $x_{new}$, 
simply check the value of its $j$th component,
and if $x_{new, j}>0$, predict 1,
otherwise predict 0.

\subsection{Number of samples needed}
???


\section{Exercise 2}

\subsection{$\hat{\bm{y}}=\bm{H}\bm{y}$}
In linear regression, we already know that
\begin{align*}
	\hat{\bm{y}} = \bm{X} \hat{\beta} = \bm{X}(\bm{X}^T\bm{X})_{-1} \bm{X}^T \bm{y}
\end{align*}
Here if we make $\bm{H} = \bm{X}(\bm{X}^T\bm{X})_{-1} \bm{X}^T$, 
then $\hat{\bm{y}}=\bm{H}\bm{y}$

In the case of kNN model, we can make a matrix $\bm{H}$ such that:
\[
	H_{i,j} = 
	\begin{cases}
		\frac{1}{k}, & \text{if } x_j \in N_k(x_i) \\
		0, & \text{otherwise}
	\end{cases}
	\text{ for } i, j \in 1,2,..., N
\]
Here $N_k(x_i)$ is the k-nearest neighborhood of observation $x_i$ 
($x_i$ itself included).
$\bm{H}$ is a $N\times N$ matrix, and each row of it only has $k$ elements with value $\frac{1}{k}$, and all other elements are 0.
With such contruction, $\hat{\bm{y}}=\bm{H}\bm{y}$.


\subsection{kNN leave-one-out}
To leave-one-out, we need to reconstruct the $\bm{H}$ matrix,
taking the sample left out into account.
Assume the $i_0$th observation needs to be left out, 
make $\bm{H}$ such that
\[
	H_{i,j} = 
	\begin{cases}
		\frac{1}{k}, & \text{if } x_j \in N_k(x_i) \text{ and } i \neq i_0 \\
		0, & \text{otherwise}
	\end{cases}
	\text{ for } i, j \in 1,2,..., N
\]
Here $N_k(x_i)$ should be the set of $x_i$'s $k$ nearest neighbors among all $x$'s but excluding $x_{i_0}$.

Denote the $\bm{H}$ matrix for $\bm{X}$ with the $i$th observation left out as $\bm{H}_i$.
The square error with the $i$th observation left out would be
\begin{align*}
	e_i & = (\hat{\bm{y}}-\bm{y})^T (\hat{\bm{y}}-\bm{y}) \\
	& = (\bm{H}_i \bm{y}-\bm{y})^T (\bm{H}_i \bm{y}-\bm{y}) \\
	& = \bm{y}^T (\bm{H}_i - \bm{I})^T (\bm{H}_i - \bm{I}) \bm{y}
\end{align*}

Leave-one-out cross validated square error would be
\begin{align*}
	SE & = \frac{1}{N} \sum_{i=1}^{N} e_i \\
	& = \frac{1}{N} \sum_{i=1}^{N}\bm{y}^T (\bm{H}_i - \bm{I})^T (\bm{H}_i - \bm{I}) \bm{y}
\end{align*}

\subsection{SVD in regression}
Plug $\bm{X} = \bm{U} \bm{D} \bm{V}^T$ into 
linear regression model $\bm{Y} = \bm{X}\beta + \bm{e} $,
we get \[ \bm{Y} = \bm{U} \bm{D} \bm{V}^T \beta + \bm{e} \]

Introduce a new vector $b = \bm{D} \bm{V}^T \beta$, 
the linear model becomes \[ \bm{Y} = \bm{U} b + \bm{e} \]

This takes the form of our normal linear regression model, 
and $b$ can be estimated by
\[ \hat{b} = (\bm{U}^T \bm{U})^{-1} \bm{U}^T  \bm{y} \]

Because $\bm{U}$ is an orthogonal matrix, $\bm{U}^T \bm{U} = \bm{I}$,
$\hat{b}$ simplifies to
\[ \hat{b} = \bm{U}^T  \bm{y} \]

And $\hat{\beta}$ can be estimated by
\[\hat{\beta} = (\bm{V}^T)^{-1} \bm{D}^{-1} \hat{b} \]

Again, because $\bm{V}$ is also an orthogonal matrix, $(\bm{V}^T)^{-1} = \bm{V}$, thus
\[\hat{\beta} = \bm{V} \bm{D}^{-1} \hat{b} \]

Define $\bm{A}=\bm{V} \bm{D}^{-1}$, we get an estimator for $\beta$ and
\[ \hat{\beta} = \bm{A} \hat{b} \]

To fit the model on a dataset, first performe SVD on the design matrix $\bm{X}$,
then calculate $\bm{D}^{-1}$ by replacing every diagonal element with its inverse (because $\bm{D}$ is diagonal, all other places shoudl be 0). Then multiply $\bm{U}$ and $\bm{y}$ to get $\hat{b}$. Finally calculate $\hat{\beta}$ via $\hat{\beta} = \bm{A} \hat{b}$.

{\scriptsize Reference: Mandel, John. ``Use of the singular value decomposition in regression analysis." The American Statistician 36.1 (1982): 15-24.}


\subsection{Change rank}
Take the first $r$ elements of $\hat{b}$ to be $\hat{b}_r$,
take the first $r$ columns of $\bm{A}$ to be $\bm{A}_r$, 
then $\hat{\beta}_r = \bm{A}_r \hat{b}_r$.

If $\bm{A}$ and $\hat{b}$ already exist,
this computation should take $O(1)$ time, 
because it is just a slicing of existing vector/matrix.
	

 
% --------------------------------------------------------------
%     You don't have to mess with anything below this line.
% --------------------------------------------------------------
 
\end{document}